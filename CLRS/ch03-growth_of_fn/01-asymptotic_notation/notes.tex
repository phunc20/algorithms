\documentclass{article}
%\usepackage[utf8]{vietnam}
\usepackage{amsmath,amssymb}
\usepackage{tikz}
 

\begin{document}

\def\firstcircle{(0,0) circle (2cm)}
\def\secondcircle{(0:2.5cm) circle (2cm)}

\section{The Big/Little-\texttt{\{oh,omega,theta\}} Notations}
Relationships btw the sets $O(g(n)),\; o(g(n)),\; \Omega(g(n)),\; \omega(g(n))\; \text{and}\; \Theta(g(n)):$ 
\vskip 1cm
  \begin{tikzpicture}
    %\filldraw[color=green, fill=green!50, very thick] \firstcircle (0,1.8) node[above,left] {$O(g(n))$};
    %\filldraw[color=red, fill=red!50, very thick] \secondcircle (2cm,1.8) node[above,right] {$\Omega(g(n))$};
    %\draw[very thick] \firstcircle (0,1.8) node[above,left,black] {$O(g(n))$};
    %\draw[very thick] \secondcircle (2cm,1.8) node[above,right,black] {$\Omega(g(n))$};
    \draw \firstcircle (-2cm,-1cm) node[above,left,black] {$O(g(n))$};
    \draw \secondcircle (4.5cm,-1cm) node[above,right,black] {$\Omega(g(n))$};
    \begin{scope}[fill opacity=0.5]
      \fill[red] \firstcircle;
      \fill[green] \secondcircle;
    \end{scope}
    \node[left] at (0,0) {$o(g(n))$};
    \node[right] at (0:2.5cm) {$\omega(g(n))$};
    \node at (0:1.25cm) {$\tiny{\Theta(g(n))}$};
  \end{tikzpicture}

  \begin{align*}
    %\Omega(g(n)) &\,\cap\, O(g(n)) \,=\, \Theta\left(g(n)\right) \\
    %o(g(n)) &\,\subset\, O(g(n)) \,-\, \Theta\left(g(n)\right)
    %\Omega(g(n)) \,\cap\, O(g(n))\; &=\, \Theta(g(n)) \\
    \Theta(g(n)) &=\, \Omega(g(n)) \,\cap\, O(g(n)) \\
    %o(g(n)) &\subset\, O(g(n)) \,-\, \Theta\left(g(n)\right) \\
    %o(g(n)) &\subsetneq\, O(g(n)) \,\setminus\, \Omega\left(g(n)\right) \\
    O(g(n)) &= o(g(n)) \sqcup \Theta(g(n)) \\
    \Omega(g(n)) &= \omega(g(n)) \sqcup \Theta(g(n)) \\
    %% redundant
    %o(g(n))      \,\cap\, \Omega(g(n)) &= \emptyset
  \end{align*}
\vskip -1em
\noindent where $\sqcup$ denotes disjoint union.
\vskip 2em

\noindent
\textbf{\textsl{3.1-1}}
Let $f(n)$ and $g(n)$ be asymptotically nonnegative functions. Using the basic definition of $\Theta$-notation,
prove that $\max\left(f(n), g(n)\right) = \Theta(f(n) + g(n))$.
\newline
\newline
%\hline

If we allow us to reframe the question. Define
\begin{align*}
	h \!:\quad n \;&\mapsto\quad \max(f(n), g(n)) \\
	l \!:\quad n \;&\mapsto\quad f(n) + g(n)
\end{align*}
Then we are supposed to show that $h(n) = \Theta(l(n))$.

\noindent
Since $f$ and $g$ are asymtotically nonnegative, we have
\begin{align*}
	& \exists\; n_1 \quad\textrm{s.t.}\quad f(n) \ge 0 \quad\forall\; n \ge n_1 \\
	& \exists\; n_2 \quad\textrm{s.t.}\quad g(n) \ge 0 \quad\forall\; n \ge n_2
\end{align*}
Let $n_3 = \max(n_1, n_2)$. Then for the constants $c_1=\frac{1}{2}, c_2=1, n_3$, we see that
\begin{align*}
	c_1 l(n) &\le h(n) \le c_2 l(n) \quad\forall\; n \ge n_3\,, \quad \textrm{i.e.}\\
	\frac{1}{2}(f(n) + g(n)) &\le \max\left(f(n), g(n)\right) \le f(n) + g(n) \quad\forall\; n \ge n_3\,.
\end{align*}
\hfill$\blacksquare$

\end{document}
